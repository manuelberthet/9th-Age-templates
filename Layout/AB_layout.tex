
%%% TO DO LIST %%%

% Alphabetical order in special rules ? -> works when no formatting.

% Placement of the different entries in the layout frame. Try to automatize it to improve space occupation.

% Placement of different profiles not reliable. Sometimes OK, sometimes one on top on of the bottom, sometimes ast the center of the page ...

% Automatic characteristics sheet at the end of the armybook ?

% Special rules categories.


\documentclass[a4paper,8pt]{extarticle} % extarticle allows to use font size of 8pt.

\usepackage[a4paper, top=1.6cm, bottom=2cm, left=1.6cm, right=1.6cm]{geometry} % Marge reduction.

%% Font and typing packages
\usepackage{fontspec}
% \setmainfont[Ligatures=TeX]{Arial} % default is Latin Modern
\newfontfamily\antiquefont[Ligatures=TeX]{Caslon Antique} % GW like font
\usepackage{microtype}			% Greatly improves general appearance of the text.
\usepackage{SIunits}			% Unit appearance.
\usepackage{xspace}				% Define commands that appear not to eat spaces.
\usepackage{ulem}				% To cross words out. Use \sout{}.

%% Array utilities
\usepackage{array}				% Additionnal options for arrays.
\usepackage{colortbl}			% Additionnal options for coloring arrays.

%% List utilities
\usepackage[inline]{enumitem}   % Display inline lists.
\usepackage{etoolbox}           % General utility. Good for lists for instance.
\usepackage{xparse}             % List utilities.
\usepackage{datatool}	% Handling alphabetical order.

%% Frames
\usepackage{framed}				% Boxes.
\usepackage[framemethod=TikZ]{mdframed}% For fancy frames.
\usepackage{tikz}				% For fancy frames.

%% Page utilities
\usepackage{multicol}			% Allows to divide a part of the page in multiple columns.
\usepackage{newfloat}			% Used to create new flottable environnements.
	\DeclareFloatingEnvironment[placement=htbp!]{unitframeFlot}
	
%% Others
\usepackage{keyval}             % Used to create maps of commands/labels/objects.
	\makeatletter                  % Mandatory for the usage of keyval.
\usepackage{xstring}            % String parsing, cutting, etc.
\usepackage[colorlinks=true]{hyperref} % Links in PDF.




%%% Command to avoid typing \xspace when creating a new name macro

\newcommand{\newnamemacro}[2]{\newcommand{#1}{#2}} % \xspace removed for compatibility with alphabetical ordering

%%% Language specific stuff

%%% Labels %%%

% Profile

\newcommand{\labels@profile}{Profile}
\newcommand{\labels@M}{M}
\newcommand{\labels@WS}{WS}
\newcommand{\labels@BS}{BS}
\newcommand{\labels@S}{S}
\newcommand{\labels@T}{T}
\newcommand{\labels@W}{W}
\newcommand{\labels@I}{I}
\newcommand{\labels@A}{A}
\newcommand{\labels@Ld}{Ld}
\newcommand{\labels@Invocation}{Invocation} % For Vampire Covenant profiles

% Technical

\newcommand{\labels@range}{Range}
\newcommand{\labels@point}{pt}
\newcommand{\labels@points}{pts}
\newcommand{\labels@only}{only}
\newcommand{\labels@magic}{Magic}
\newcommand{\labels@pathsused}{Generate spells from Path of}
\newcommand{\labels@additionalmodels}{additional models}

% Unit entry labels

\newcommand{\labels@unitsize}{Unit size}
\newcommand{\labels@basesize}{Base size}
\newcommand{\labels@trooptype}{Troop type}
\newcommand{\labels@specialrules}{Special rules}
\newcommand{\labels@equipment}{Equipment}
\newcommand{\labels@options}{Options}
\newcommand{\labels@commandgroup}{Command Group}
\newcommand{\labels@mounts}{Mounts}
\newcommand{\labels@specialequipment}{Special Equipment}

% Command groups

\newcommand{\labels@champion}{Champion}
\newcommand{\labels@standardbearer}{Standard Bearer}
\newcommand{\labels@musician}{Musician}
\newcommand{\labels@singlebannerallowance}{One unit may take a Magic banner}
\newcommand{\labels@condsinglebannerallowance}{One unit may take a Magic banner if}
\newcommand{\labels@bannerallowance}{May take a Magic banner}
\newcommand{\labels@championallowance}{May take Magic items}

% Titles

\newcommand{\labels@lords}{Lords}
\newcommand{\labels@heroes}{Heroes}
\newcommand{\labels@coreunits}{Core units}
\newcommand{\labels@specialunits}{Special units}
\newcommand{\labels@rareunits}{Rare units}
\newcommand{\labels@armyspecialrules}{Army special rules}
\newcommand{\labels@armoury}{Armoury}
\newcommand{\labels@magicitems}{Magic items}
\newcommand{\labels@magicweapons}{Magic weapons}
\newcommand{\labels@magicarmor}{Magic armor}
\newcommand{\labels@talismans}{Talismans}
\newcommand{\labels@enchanteditems}{Enchanted items}
\newcommand{\labels@arcaneitems}{Arcane items}
\newcommand{\labels@magicbanners}{Magic banners}

% Titlepage

\newcommand{\labels@fantasybattles}{Fantasy Battles}
\newcommand{\labels@NinthAge}{The 9th Age}
\newcommand{\labels@creators}{A collaboration between ETC and Swedish Comp System}
\newcommand{\labels@latexcredit}{Layout designed using \LaTeX .}


%%% Technical commands

\newcommand{\free}{free}
\newcommand{\upto}{up to}
\newcommand{\Upto}{Up to}
\newcommand{\unlimited}{unlimited}
\newcommand{\permodel}{/model}
\newcommand{\listlastchoice}{, or}
\newcommand{\notif}[1]{(not if #1)}
\newcommand{\wordand}{and}


%%% Special rules %%%

\newnamemacro{\ambush}{\specialrule{Ambush}}
\newcommand{\armourpiercing}[1]{\specialrule{Armour Piercing\ifblank{#1}{}{~(#1)}}}
\newnamemacro{\blurry}{\specialrule{Blurry}}
\newcommand{\bodyguard}[1]{\specialrule{Bodyguard\ifblank{#1}{}{~(#1)}}}
\newcommand{\breathweapon}[1]{\specialrule{Breath Weapon\ifblank{#1}{}{ (#1)}}}
\newnamemacro{\channel}{\specialrule{Channel}}
\newnamemacro{\crushattack}{\specialrule{Crush Attack}}
\newnamemacro{\devastatingcharge}{\specialrule{Devastating Charge}}
\newnamemacro{\distracting}{\specialrule{Distracting}}
\newnamemacro{\engineer}{\specialrule{Engineer}}
\newnamemacro{\ethereal}{\specialrule{Ethereal}}
\newnamemacro{\fastcavalry}{\specialrule{Fast Cavalry}}
\newnamemacro{\fear}{\specialrule{Fear}}
\newnamemacro{\fightinextrarank}{\specialrule{Fight in Extra Rank}}
\newnamemacro{\fireborn}{\specialrule{Fireborn}}
\newnamemacro{\flamingattacks}{\specialrule{Flaming Attacks}}
\newnamemacro{\flammable}{\specialrule{Flammable}}
\newnamemacro{\freereform}{\specialrule{Free Reform}}
\newnamemacro{\frenzy}{\specialrule{Frenzy}}
\newcommand{\fly}[1]{\specialrule{Fly\ifblank{#1}{}{~(#1)}}}
\newcommand{\grindingattacks}[1]{\specialrule{Grinding Attacks\ifblank{#1}{}{~(#1)}}}
\newnamemacro{\hatred}{\specialrule{Hatred}}
\newnamemacro{\hellfire}{\specialrule{Hellfire}}
\newnamemacro{\hidden}{\specialrule{Hidden}}
\newnamemacro{\holyattacks}{\specialrule{Holy Attacks}}
\newnamemacro{\immunetopsychology}{\specialrule{Immune to Psychology}}
\newcommand{\impacthits}[1]{\specialrule{Impact Hits\ifblank{#1}{}{~(#1)}}}
\newnamemacro{\insignificant}{\specialrule{Insignificant}}
\newnamemacro{\largetarget}{\specialrule{Large Target}}
\newnamemacro{\lethalstrike}{\specialrule{Lethal Strike}}
\newnamemacro{\lightningattacks}{\specialrule{Lightning Attacks}}
\newnamemacro{\lightningreflexes}{\specialrule{Lightning Reflexes}}
\newcommand{\magicresistance}[1]{\specialrule{Magic Resistance\ifblank{#1}{}{~(#1)}}}
\newnamemacro{\magicalattacks}{\specialrule{Magical Attacks}}
\newnamemacro{\metalshifting}{\specialrule{Metalshifting}}
\newnamemacro{\moveorfire}{\specialrule{Move or Fire}}
\newcommand{\multipleshots}[1]{\specialrule{Multiple Shots\ifblank{#1}{}{ (#1)}}}
\newcommand{\multiplewounds}[2]{\specialrule{Multiple Wounds\ifblank{#1}{}{ (#1\ifblank{#2}{)}{, #2)}}}}
\newnamemacro{\notaleader}{\specialrule{Not a Leader}}
\newnamemacro{\otherworldly}{\specialrule{Otherworldly}}
\newcommand{\pathmaster}[1]{\specialrule{Pathmaster\ifblank{#1}{}{ (#1)}}}
\newnamemacro{\poisonedattacks}{\specialrule{Poisoned Attacks}}
\newnamemacro{\quicktofire}{\specialrule{Quick to Fire}}
\newcommand{\randommovement}[1]{\specialrule{Random Movement\ifblank{#1}{}{~(#1)}}}
\newcommand{\randomattacks}[1]{\specialrule{Random Attacks\ifblank{#1}{}{~(#1)}}}
\newcommand{\regeneration}[1]{\specialrule{Regeneration\ifblank{#1}{}{ (#1+)}}}
\newnamemacro{\requirestwohands}{\specialrule{Requires Two Hands}}
\newnamemacro{\scythes}{\specialrule{Scythes}}
\newnamemacro{\scout}{\specialrule{Scout}}
\newnamemacro{\scouts}{\specialrule{Scouts}}
\newnamemacro{\slowtofire}{\specialrule{Slow to Fire}}
\newcommand{\stomp}[1]{\specialrule{Stomp\ifblank{#1}{}{~(#1)}}}
\newcommand{\strider}[1]{\specialrule{Strider\ifblank{#1}{}{~(#1)}}}
\newnamemacro{\stubborn}{\specialrule{Stubborn}}
\newnamemacro{\stupidity}{\specialrule{Stupidity}}
\newnamemacro{\skirmisher}{\specialrule{Skirmisher}}
\newnamemacro{\skirmishers}{\specialrule{Skirmishers}}
\newnamemacro{\swiftstride}{\specialrule{Swiftstride}}
\newnamemacro{\thunderouscharge}{\specialrule{Thunderous Charge}}
\newnamemacro{\terror}{\specialrule{Terror}}
\newnamemacro{\toxicattacks}{\specialrule{Toxic Attacks}}
\newnamemacro{\unbreakable}{\specialrule{Unbreakable}}
\newnamemacro{\undead}{\specialrule{Undead}}
\newnamemacro{\unstable}{\specialrule{Unstable}}
\newnamemacro{\unwieldy}{\specialrule{Unwieldy}}
\newnamemacro{\vanguard}{\specialrule{Vanguard}}
\newnamemacro{\volleyfire}{\specialrule{Volley Fire}}
\newnamemacro{\warplatform}{\specialrule{War Platform}}
\newcommand{\wardsave}[1]{\specialrule{Ward Save\ifblank{#1}{}{~(#1+)}}}
\newnamemacro{\weaponmaster}{\specialrule{Weapon Master}}
\newcommand{\wizardconclave}[1]{\specialrule{Wizard Conclave\ifblank{#1}{}{ (#1)}}}


%%% Magic %%%

\newnamemacro\battle{Battle}
\newnamemacro\alchemy{Alchemy}
\newnamemacro\death{Death}
\newnamemacro\fire{Fire}
\newnamemacro\heavens{Heavens}
\newnamemacro\light{Light}
\newnamemacro\nature{Nature}
\newnamemacro\shadows{Shadows}
\newnamemacro\wilderness{Wilderness}
\newnamemacro\butchery{Butchery}
\newnamemacro\change{Change}
\newnamemacro\thebiggreengods{the Big Green Gods}
\newnamemacro\thelittlegreengods{the Little Green Gods}
\newnamemacro\blackmagic{Black Magic}
\newnamemacro\disease{Disease}
\newnamemacro\lust{Lust}
\newnamemacro\necromancy{Necromancy}
\newnamemacro\ruin{Ruin}
\newnamemacro\forge{the Forge}
\newnamemacro\sands{the Sands}
\newnamemacro\whitemagic{White Magic}

\newcommand{\magiclevel}[1]{Level #1 \wizard}

\newnamemacro{\wizard}{Wizard}

\newcommand{\boundspell}[1]{Bound Spell, Power Level #1}


%%% Other rules %%%

\newnamemacro{\armoursave}{Armour Save}
\newnamemacro{\firstinrank}{First in Rank}
\newnamemacro{\hardcover}{Hard Cover}
\newnamemacro{\holdyourground}{\specialrule{Hold your Ground}}
\newnamemacro{\inspiringpresence}{\specialrule{Inspiring Presence}}
\newnamemacro{\lightcover}{Light Cover}
\newnamemacro{\monstrousrank}{Monstrous Rank}
\newnamemacro{\oneofakind}{One of a kind}
\newnamemacro{\ordnance}{Ordnance}
\newnamemacro{\parry}{Parry}
\newnamemacro{\raisewounds}{Raise Wounds}
\newnamemacro{\recoverwounds}{Recover Wounds}
\newnamemacro{\aideddispel}{Aided Dispel}
\newnamemacro{\rnf}{R\&{}F}


%%% Equipment %%%

\newcommand{\innatedefence}[1]{Innate Defence\ifblank{#1}{}{~(#1+)}}
\newcommand{\mountsprotection}[1]{Mount's Protection\ifblank{#1}{}{~(#1+)}}
\newnamemacro{\la}{Light Armour}
\newnamemacro{\ha}{Heavy Armour}
\newnamemacro{\platearmour}{Plate Armour}
\newnamemacro{\hw}{Hand Weapon}
\newnamemacro{\ahw}{Additional Hand Weapon}
\newnamemacro{\spear}{Spear}
\newnamemacro{\halberd}{Halberd}
\newnamemacro{\gw}{Great Weapon}
\newnamemacro{\lance}{Lance}
\newnamemacro{\lightlance}{Light Lance}
\newnamemacro{\shield}{Shield}
\newnamemacro{\barding}{Barding}
\newnamemacro{\throwingweapons}{Throwing Weapons}

\newnamemacro{\boltthrower}{Bolt Thrower}
\newnamemacro{\artilleryweapon}{Artillery Weapon}


%%% Troop types %%%

\newnamemacro{\infantry}{Infantry}
\newnamemacro{\monstrousinfantry}{Monstrous Infantry}
\newnamemacro{\cavalry}{Cavalry}
\newnamemacro{\monstrouscavalry}{Monstrous Cavalry}
\newnamemacro{\swarm}{Swarm}
\newnamemacro{\swarms}{Swarms}
\newnamemacro{\warbeast}{War Beast}
\newnamemacro{\warbeasts}{War Beasts}
\newnamemacro{\monster}{Monster}
\newnamemacro{\monsters}{Monsters}
\newnamemacro{\monstrousbeast}{Monstrous Beast}
\newnamemacro{\monstrousbeasts}{Monstrous Beasts}
\newnamemacro{\chariot}{Chariot}
\newnamemacro{\chariots}{Chariots}
\newnamemacro{\riddenmonster}{Ridden Monster}
\newnamemacro{\riddenmonsters}{Ridden Monsters}


%%% Profile wording

\newcommand{\magicitemsallowance}{May take Magic Items}
\newcommand{\anyofthefollowing}{\optionschoice{May take any of the following:}}
\newcommand{\weapononechoice}{\optionschoice{May take a weapon (one choice only):}}
\newcommand{\magiclevelchoice}{\optionschoice{May become one of the following:}}
\newcommand{\bsboption}{May be the Battle Standard Bearer}

%%% Commands to handle strings, better than xstring to handle commands inside the strings %%%

\newcommand{\splitatstar}[3]{%
  \protected@edef\split@temp{#1}%
  \saveexpandmode
  \expandarg\StrCut{\split@temp}{*}#2#3%
  \restoreexpandmode
}

\newcommand{\splitatinf}[3]{%
  \protected@edef\split@temp{#1}%
  \saveexpandmode
  \expandarg\StrCut{\split@temp}{<}#2#3%
  \restoreexpandmode
}

\newcommand{\splitatequal}[3]{%
  \protected@edef\split@temp{#1}%
  \saveexpandmode
  \expandarg\StrCut{\split@temp}{=}#2#3%
  \restoreexpandmode
}

\newcommand{\ifsubstring}[4]{%
  \protected@edef\split@temp{#1}%
  \protected@edef\split@tempbis{#2}%
  \saveexpandmode
  \expandarg\IfSubStr{\split@temp}{\split@tempbis}{#3}{#4}%
  \restoreexpandmode
}


%%% Commands for alphabetical order sorting %%%

\newcommand{\sortitem}[2][\relax]{%
  \DTLnewrow{list}% Create a new entry
  \ifx#1\relax
    \DTLnewdbentry{list}{sortlabel}{#2}% Add entry sortlabel (no optional argument)
  \else
    \DTLnewdbentry{list}{sortlabel}{#1}% Add entry sortlabel (optional argument)
  \fi%
  \DTLnewdbentry{list}{description}{#2}% Add entry description
}
\newenvironment{sortedlist}{%
  \DTLifdbexists{list}{\DTLcleardb{list}}{\DTLnewdb{list}}% Create new/discard old list
}{%
  \DTLsort{sortlabel}{list}% Sort list
  \begin{itemize*}[label={}, itemjoin={,}]%
    \DTLforeach*{list}{\theDesc=description}{%
      \item\theDesc}% Print each item
  \end{itemize*}%
}

\pdfstringdefDisableCommands{\def\textcolor#1{}}

\newcommand{\addtosortedlist}[1]{%
	\pdfstringdef\plaintexttemp{#1}%
	\expandafter\sortitem\expandafter[\plaintexttemp]{#1}%
}%


%%% Technical commands %%%

\newcommand{\newrule}{\textcolor{green!50!black}}
\newcommand{\removedrule}[1]{\textcolor{green!50!black}{\sout{#1}}}

\newcommand{\inch}{\arcsecond}
\newcommand{\foot}{\arcminute}
\newcommand{\range}[1] {\labels@range~\unit{#1}{\inch}}
\newcommand{\distance}[1] {\unit{#1}{\inch}}
\newcommand{\result}[1] {\texttt{'}$ #1 $\texttt{'}}
\newcommand{\pts}[1]{\unit{#1}{\expandafter\ifstrequal\expandafter{#1}{1}{\labels@point}{\labels@points}}}


%%% Titles %%%

\newcommand{\armytitle}[1]{\noindent\begin{center}\Huge{\textbf{\antiquefont\expandafter\uppercase\expandafter{#1}}}\end{center}\medskip}
\newcommand{\lordstitle}{\def\logolocalpath{../Layout/pics/logo_lord.png}\armytitle{\labels@lords}}
\newcommand{\heroestitle}{\def\logolocalpath{../Layout/pics/logo_hero.png}\clearpage\armytitle{\labels@heroes}}
\newcommand{\coreunitstitle}{\def\logolocalpath{../Layout/pics/logo_core.png}\clearpage\armytitle{\labels@coreunits}}
\newcommand{\specialunitstitle}{\def\logolocalpath{../Layout/pics/logo_special.png}\clearpage\armytitle{\labels@specialunits}}
\newcommand{\rareunitstitle}{\def\logolocalpath{../Layout/pics/logo_rare.png}\clearpage\armytitle{\labels@rareunits}}
\newcommand{\mountstitle}{\def\logolocalpath{../Layout/pics/logo_mount.png}\clearpage\armytitle{\labels@mounts}}

\newcommand{\armyspecialrules}{\newpage\twocolumn\noindent\begin{center}\Huge{\textbf{\antiquefont\labels@armyspecialrules}}\end{center}}
\newcommand{\armyspecialruleentry}[1]{\subsection*{\LARGE\antiquefont\textit{#1}}}

\newcommand{\armyarmoury}{\newpage\noindent\begin{center}\Huge{\textbf{\antiquefont\labels@armoury}}\end{center}}

\newcommand{\armymagicitems}{\newpage\noindent\begin{center}\Huge{\textbf{\antiquefont\labels@magicitems}}\end{center}}
\newcommand{\armymagicweapons}{\subsection*{\LARGE\antiquefont\labels@magicweapons}}
\newcommand{\armymagicarmor}{\subsection*{\LARGE\antiquefont\labels@magicarmor}}
\newcommand{\armytalismans}{\subsection*{\LARGE\antiquefont\labels@talismans}}
\newcommand{\armyenchanteditems}{\subsection*{\LARGE\antiquefont\labels@enchanteditems}}
\newcommand{\armyarcaneitems}{\subsection*{\LARGE\antiquefont\labels@arcaneitems}}
\newcommand{\armymagicbanners}{\subsection*{\LARGE\antiquefont\labels@magicbanners}}

\newcommand{\armynewsection}[1]{\newpage\noindent\begin{center}\Huge{\textbf{\antiquefont#1}}\end{center}\vspace*{0.2cm}}

\newcommand{\armynewsubsection}[1]{\subsection*{\LARGE\antiquefont#1}}
\newcommand{\armynewsubsubsection}[1]{\subsubsection*{\antiquefont#1}}

\newcommand{\armylist}{\clearpage\onecolumn}

\newcommand{\separator}{\noindent\textcolor{black!30}{\rule{\columnwidth}{1pt}}}


%%% Custom lists and description for first sections of the army books

\newcommand{\startpricelist}{\begin{description}[leftmargin=0.3cm, labelindent=0cm, labelsep=0.1cm]}
\def\endpricelist{\end{description}}
\newcommand{\pricelistitem}[2]{\item \option{\textbf{#1}}{#2}\newline}

\newenvironment{customitemize}{\begin{description}[leftmargin=0.3cm, labelindent=0cm, labelsep=0cm]}{\end{description}}
\newenvironment{customsubitemize}{\begin{itemize}[label={-}, labelsep=0.1cm, topsep=0cm, parsep=0cm, itemsep=0cm, leftmargin=0.4cm, labelindent=0cm]}{\end{itemize}}


%%% Table parameters %%%

\newcolumntype{M}[1]{>{\centering\let\newline\\\arraybackslash\hspace{0pt}}m{#1}}


%%%  Lists handling %%%

\newcommand{\addlocallist}{\listadd\locallists@dummy}
\NewDocumentCommand{\parsespacelist}{ >{\SplitList{ }} m } {%
	\ProcessList{#1}{\addlocallist}%
}
\NewDocumentCommand{\parsecommalist}{ >{\SplitList{,}} m } {%
	\ProcessList{#1}{\addlocallist}%
}
\newcommand{\parselist}[3][,]{%
	\renewcommand\addlocallist{\listadd#3}%
  	\undef#3%
  	\ifstrequal{#1}{ }{\parsespacelist{#2}}{\parsecommalist{#2}}%
}


%%% Profiles handling %%%

% Element of a table that contains the characteristics of a model (or part of a model)
\newcommand\caraclist[1]{
	\parselist[ ]{#1}{\locallists@caraclist}%
	\forlistloop{&}{\locallists@caraclist}%
}

% Line of a profile table, including bottom line. It is meant to contain the name of the model (or part), its characteristics (preferably, the second argument should contain the \carac macro), troop type and base size.
\newcommand{\profilefirstline}[4]{#1 & #2 &   & #3 & #4 }

% Start of a profile table. Includes the table commands, and the column labels. \profilecellsize is the size of the characteristics cells in the profile.
\newcommand{\profilecellsize}{0.50cm}
\newcommand{\profilestart}{%
	\noindent %
	\begin{tabular}{@{}p{4cm}@{}M{\profilecellsize}@{}M{\profilecellsize}@{}M{\profilecellsize}@{}M{\profilecellsize}@{}M{\profilecellsize}@{}M{\profilecellsize}@{}M{\profilecellsize}@{}M{\profilecellsize}@{}M{\profilecellsize}@{}p{1cm}@{}p{3.8cm}@{}p{2cm}@{}}%
	\textbf{\labels@profile} &%
	\labels@M & \labels@WS & \labels@BS & \labels@S & \labels@T & \labels@W & \labels@I & \labels@A & \labels@Ld &%
	&%
	\textbf{\labels@trooptype} &%
	\textbf{\labels@basesize}%
}

% End of a profile table.
\newcommand{\profileend}{\end{tabular}}

% Algorithm to automatically use and fill previous command, with coherence check.
\providebool{profilefirst}
\newcommand{\profileitem}[1]{%
	\tabularnewline%
	\splitatinf{#1}\local@unitname\local@unitprofile%
	\local@unitname \expandafter\caraclist\expandafter{\local@unitprofile}%
	&%
	& \ifbool{profilefirst}{\unit@type}{}%
	& \ifbool{profilefirst}{\unit{\unit@basesize}{\milli\meter}}{}%
	\global\boolfalse{profilefirst}%
}
\newcommand{\profile}[1]{%
	\parselist{#1}{\locallists@profileslist}%
	\profilestart%
	\global\booltrue{profilefirst}%
	\forlistloop{\profileitem}{\locallists@profileslist}%
	\profileend%
}


%%%%%%%%%%%%%%%%%%
%%% Unit rules %%%
%%%%%%%%%%%%%%%%%%

%%% Entry title command %%%

\newcommand{\unitentry}[2]{\ifdefempty{#1}{}{\noindent #2 \medskip}}
\newcommand{\unitentrynoskip}[2]{\ifdefempty{#1}{}{\noindent #2}}


%%% Unit size %%%

\newcommand{\unitsize}[1]{\unitentry{#1}{\textbf{\labels@unitsize :} #1}}


%%% Special rules %%%

% Formatting for a special rule. Currently italicized.
\newcommand{\specialrule}[1]{\textit{#1}}
\newcommand{\only}[1]{\textnormal{(#1 \labels@only)}}

% Special rules listing for a unit, with alphabetical order.
\newcommand{\ruleslist}[1]{%
	\parselist[,]{#1}{\locallists@ruleslist}%
	\begin{sortedlist}%
		\forlistloop{\addtosortedlist}{\locallists@ruleslist}%
	\end{sortedlist}%
}

% Special rules entry.
\newcommand{\specialrules}[1]{\unitentry{#1}{\textbf{\labels@specialrules :}\expandafter\ruleslist\expandafter{#1}.}}


%%% Magical abilities %%%

% Paths listing for a unit.
\newcommand{\pathslist}[1]{%
	\parselist[,]{#1}{\locallists@pathslist}%
	\begin{itemize*}[label={}, itemjoin={,}, itemjoin*={\listlastchoice}]%
		\forlistloop{\item}{\locallists@pathslist}%
	\end{itemize*}%
}

% Magic entry.
\newcommand{\magic}[2]{\unitentry{#2}{\textbf{\labels@magic : }\ifdefempty{#1}{}{\magiclevel{#1}. }\labels@pathsused\expandafter\pathslist\expandafter{#2}.}}


%%% Equipment %%%

% Equipment listing.
\newcommand{\equipmentlist}[1]{%
	\parselist[,]{#1}{\locallists@equipmentlist}%
	\begin{itemize*}[label={}, itemjoin={,}]%
	\forlistloop{\item}{\locallists@equipmentlist}%
	\end{itemize*}%
}

% Equipment entry.
\newcommand{\equipment}[1]{\unitentry{#1}{\textbf{\labels@equipment :}\expandafter\equipmentlist\expandafter{#1}.}}


%%% Options %%%

% Frame commands.
\newcommand{\optionsframestart}{\begin{innerframe}[\labels@options]}
\newcommand{\optionsframeend}{\end{innerframe}}

% Options listing.
\newcommand{\optionslist}[1]{%
	\parselist[,]{#1}{\locallists@optionslist}%
	\begin{description}[leftmargin=0.3cm, labelindent=0cm, labelsep=0cm, itemsep=0cm, parsep=0cm]%
		\forlistloop{\item\setoption}{\locallists@optionslist}%
	\end{description}%
}

% Options entry.
\newcommand{\options}[1]{\ifdefempty{#1}{}{\optionsframestart\vspace*{-0.4cm}\unitentrynoskip{#1}{\expandafter\optionslist\expandafter{#1}}\optionsframeend}}

% Option specific commands.
\newcommand{\setoption}[1]{%
	\noexpandarg\StrCut{#1}{=}\optiontext\optionvalue%
	\expandafter\ifstrequal\expandafter{\optionvalue}{}{%
		\optiontext%
	}{%
	\ifsubstring{\optionvalue}{\free}{%
		\option[\free]{\optiontext}{\optionvalue}%
	}{%
	\ifsubstring{\optionvalue}{\unlimited}{%
		\option[\unlimited]{\optiontext}{\optionvalue}%
	}{%
	\ifsubstring{\optionvalue}{\upto}{%
		\splitatinf{\optionvalue}\myoption\myvalue%
		\option[\upto]{\optiontext}{\myvalue}%
	}{%
	\ifsubstring{\optionvalue}{\permodel}{%
		\splitatinf{\optionvalue}\myoption\myvalue%
		\option[\permodel]{\optiontext}{\myvalue}%
	}{%
		\option{\optiontext}{\optionvalue}%
	}}}}}%
}

\newcommand{\option}[3][]{#2\dotfill%
	% Add \upto token if necessary.
	\ifstrequal{#1}{\upto}{\upto~}{}%
	% The option can be free, have an unlimited cost, or have a points cost.
	\ifstrequal{#1}{\free}{\free}{\ifstrequal{#1}{\unlimited}{\unlimited}{\pts{#3}}}%
	% Add \permodel if necessary.
	\ifstrequal{#1}{\permodel}{\permodel}{}%
}
\newcommand\optionschoice[2]{%
	\parselist[,]{#2}{\locallists@optionschoice}%
	#1%
	\begin{itemize}[label={}, parsep=0cm, labelindent=0cm, labelwidth=0cm, noitemsep, topsep=0em, leftmargin=0.3cm]%
	\forlistloop{\item\setoption}{\locallists@optionschoice}%
	\end{itemize}%
}

% Option description in army desc.
\newcommand{\optiondef}[3]{\option{\textbf{#1}}{#2}\ifblank{#3}{}{\\{#3}}}


%%% Mount options %%%

% Frame commands.
\newcommand{\mountsframestart}{\begin{innerframe}[\labels@mounts]}
\newcommand{\mountsframeend}{\end{innerframe}}

% Mount listing.
\newcommand{\mountslist}[1]{%
	\parselist[,]{#1}{\locallists@mountslist}%
	\begin{description}[leftmargin=0.3cm, labelindent=0cm, labelsep=0cm, itemsep=0cm, parsep=0cm]%
		\forlistloop{\item\setmount}{\locallists@mountslist}%
	\end{description}%
}

% Mount specific command.
\newcommand{\setmount}[1]{%
	\splitatequal{#1}\mountname\mountvalue%
	\expandafter\ifstrequal\expandafter{\mountvalue}{}%
		{\mountname}%
		{\option{\mountname}{\mountvalue}}%
}

% Mount entry.
\newcommand{\mounts}[1]{\ifdefempty{#1}{}{\mountsframestart\vspace*{-0.4cm}\unitentrynoskip{#1}{\expandafter\mountslist\expandafter{#1}}\mountsframeend}}


%%% Command group %%%

% Command group specific commands.
\define@key{commandgroup}{restriction}            {\def\commandgroup@restriction{#1}}
\define@key{commandgroup}{champion}               {\def\commandgroup@champion{#1}}
\define@key{commandgroup}{championallowance}      {\def\commandgroup@championallowance{#1}}
\define@key{commandgroup}{championoption}         {\def\commandgroup@championoption{#1}}
\define@key{commandgroup}{championrestriction}    {\def\commandgroup@championrestriction{#1}}
\define@key{commandgroup}{banner}                 {\def\commandgroup@banner{#1}}
\define@key{commandgroup}{bannerallowance}        {\def\commandgroup@bannerallowance{#1}}
\define@key{commandgroup}{singlebannerallowance}  {\def\commandgroup@singlebannerallowance{#1}}
\define@key{commandgroup}{condsinglebannerallowance}  {\def\commandgroup@condsinglebannerallowance{#1}}
\define@key{commandgroup}{banneroption}           {\def\commandgroup@banneroption{#1}}
\define@key{commandgroup}{bannerrestriction}      {\def\commandgroup@bannerrestriction{#1}}
\define@key{commandgroup}{musician}               {\def\commandgroup@musician{#1}}
\define@key{commandgroup}{musicianrestriction}    {\def\commandgroup@musicianrestriction{#1}}
\newcommand{\defcommandgroup}{%
	\setkeys{commandgroup}{restriction=,
	                       champion=, championallowance=, championoption=, championrestriction=,
	                       banner=, bannerallowance=, singlebannerallowance=, condsinglebannerallowance=, banneroption=, bannerrestriction=,
	                       musician=, musicianrestriction=}%
	\setkeys{commandgroup}%
}

% Frame commands.
\newcommand{\commandgroupframestart}{\begin{innerframe}[\labels@commandgroup]}
\newcommand{\commandgroupframeend}{\end{innerframe}}

% Command group entry.
\newcommand{\commandgroup}[1]{%
	\defcommandgroup{#1}%
	\ifstrempty{#1}{}{\commandgroupframestart\vspace*{-0.2cm}%
		\begin{description}[leftmargin=0.3cm, labelindent=0cm, labelsep=0cm, itemsep=0cm, parsep=0cm]%
			% Command group title, including restrictions applying to all the command group
			\item \textbf{\expandafter\ifblank\expandafter{\commandgroup@restriction}{}{ \only{\commandgroup@restriction}~: }} 
			% Champion handling.
			\ifdefempty{\commandgroup@champion}{}{% We have a champion!
				\item \hspace*{-0.04cm}\option{\labels@champion%
					% Possible restrictions to taking a champion
				    \expandafter\ifblank\expandafter{\commandgroup@championrestriction}{}{ \only{\commandgroup@championrestriction}}%
				    % Cost of a champion
				    }{\commandgroup@champion}%
				    % Magical allowance of the champion. Should probably not be used, champion option can do it as well and is more flexible.
					\ifdefempty{\commandgroup@championallowance}{}{\par\option[\upto]{vspace{0.3cm}- \labels@championallowance}{\commandgroup@championallowance}}%
					% Any option available to the champion, in the form option:cost
					\ifdefempty{\commandgroup@championoption}{}{%
						\splitatinf{\commandgroup@championoption}\local@option\local@cost%
						\par\option{\hspace*{0.3cm}- \local@option}{\local@cost}}%
			}% End of champion handling
			\ifdefempty{\commandgroup@banner}{}{% We have a banner!
				\item \hspace*{-0.04cm}\option{\labels@standardbearer%
					% Possible restrictions to taking a banner
				    \expandafter\ifblank\expandafter{\commandgroup@bannerrestriction}{}{ \only{\commandgroup@bannerrestriction}}%
				    % Cost of a banner
				    }{\commandgroup@banner}%
				    % Magical banner, if all units of this type can take one.
					\ifdefempty{\commandgroup@bannerallowance}{}{\par\option[\upto]{\hspace*{0.3cm}- \labels@bannerallowance}{\commandgroup@bannerallowance}}%
					% Magical banner, if only one unit of this type can take one.
					\ifdefempty{\commandgroup@singlebannerallowance}{}{\par\option[\upto]{\hspace*{0.3cm}- \labels@singlebannerallowance}{\commandgroup@singlebannerallowance}}%
					% Magical banner, if only one unit of this type can take one, but with condtions.
					\ifdefempty{\commandgroup@condsinglebannerallowance}{}{%
						\splitatinf{\commandgroup@condsinglebannerallowance}\local@option\local@cost%
						\par\option[\upto]{\hspace*{0.3cm}- \labels@condsinglebannerallowance \local@option}{\local@cost}}%
					% Additional option for the banner, such as Hill Goblin Lookouts for Ogres
					\ifdefempty{\commandgroup@banneroption}{}{
						\splitatinf{\commandgroup@banneroption}{\local@option}{\local@cost} 
						\par\option{\hspace*{0.3cm}- \local@option}{\local@cost}
					}%
			}%
			\ifdefempty{\commandgroup@musician}{}{% We have a musician!
				\item \hspace*{-0.04cm}\option{\labels@musician%
					% Possible restrictions to taking a musician
				    \expandafter\ifblank\expandafter{\commandgroup@musicianrestriction}{}{ \only{\commandgroup@musicianrestriction}}%
				    % Cost of a musician
				    }{\commandgroup@musician}%
			}%
		\end{description}%
	\commandgroupframeend%
	 }%
}


%%% Unit rules %%%

% Frame commands.
\newcommand{\unitrulesframestart}{\begin{innerframe}[\labels@specialrules]}
\newcommand{\unitrulesframeend}{\end{innerframe}}

% Unit rules specific commands.
\newcommand{\unitrule}[2]{\item[#1~:]#2}

% Unit rule entry.
\newcommand{\unitrules}[1]{\ifdefempty{#1}{}{\unitrulesframestart\vspace*{-0.05cm}\begin{description}[leftmargin=0.3cm, labelindent=0cm, labelsep=0.1cm, itemsep=0cm, parsep=0cm]#1\end{description}\unitrulesframeend}}


%%% Special equipment %%%

% Frame commands.
\newcommand{\unitequipmentframestart}{\begin{innerframe}[\labels@specialequipment]}
\newcommand{\unitequipmentframeend}{\end{innerframe}}

% Special equipment specific commands.
\newcommand{\equipmentdef}[2]{\item[#1~:]#2}

% Special equipment entry.
\newcommand{\unitequipment}[1]{\ifdefempty{#1}{}{\unitequipmentframestart\vspace*{-0.05cm}\begin{description}[leftmargin=0.3cm, labelindent=0cm, labelsep=0.1cm, itemsep=0cm, parsep=0cm]#1\end{description}\unitequipmentframeend}}






%%%%%%%%%%%%%%%%%%%%%%%%%%%%%%%%
%%% Profile input and layout %%%
%%%%%%%%%%%%%%%%%%%%%%%%%%%%%%%%

%%% Input parameters %%%

\define@key{unit}{name}{\def\unit@name{#1}}
\define@key{unit}{profile}{\def\unit@profile{#1}}
\define@key{unit}{cost}{\def\unit@cost{#1}}
\define@key{unit}{costpermodel}{\def\unit@costpermodel{#1}}
\define@key{unit}{additionalmodels}{\def\unit@additionalmodels{#1}}
\define@key{unit}{type}{\def\unit@type{#1}}
\define@key{unit}{unitsize}{\def\unit@unitsize{#1}}
\define@key{unit}{basesize}{\def\unit@basesize{#1}}
\define@key{unit}{specialrules}{\def\unit@specialrules{#1}}
\define@key{unit}{magiclevel}{\def\unit@magiclevel{#1}}
\define@key{unit}{magicpaths}{\def\unit@magicpaths{#1}}
\define@key{unit}{equipment}{\def\unit@equipment{#1}}
\define@key{unit}{unitequipment}{\def\unit@unitequipment{#1}}
\define@key{unit}{options}{\def\unit@options{#1}}
\define@key{unit}{mounts}{\def\unit@mounts{#1}}
\define@key{unit}{commandgroup}{\def\unit@commandgroup{#1}}
\define@key{unit}{unitrules}{\def\unit@unitrules{#1}}
\define@key{unit}{additional}{\def\unit@additional{#1}}


%%% Frames definition %%%

% Unit's big frame.
\tikzset{unitprice/.style={draw=black, fill=white, rectangle, rounded corners, right, minimum height=0.7cm, font=\bfseries}}
\tikzset{unittitle/.style={draw=black, fill=white, rectangle, rounded corners, right, minimum height=0.7cm, font=\bfseries}}
\tikzset{unitlogo/.style={draw=white, fill=white, rectangle, right, minimum height=0.7cm}}

\newenvironment{unitframe}[2][]{%
	\mdfsetup{%
	          linewidth=1pt,%
	          roundcorner=5pt,%
	          backgroundcolor=white,%
	          innertopmargin=1.2\baselineskip,
	          innerbottommargin=1.2\baselineskip,
			  singleextra={
				\node[unitprice,anchor=east,xshift=-0.5cm] at (P)%
					{\large\pts{\unit@cost}};
				\node[unittitle,xshift=0.3cm] at (P-|O)%
					 {\LARGE\antiquefont\uppercase\expandafter\expandafter\expandafter{\unit@name}};
				\node[unitlogo, xshift=8.11cm, yshift=0.1cm] at (P-|O)%
					{\includegraphics[width=1.2cm]{\logolocalpath}};
			  }
	}%
	\begin{mdframed}[]\relax%
}%
{%
\end{mdframed}%
}

% Inner small frames for options, special rules definition, ...
\tikzset{innertitle/.style={fill=white, rectangle, rounded corners, right, minimum height=8pt, font=\bfseries, xshift=0.5cm}}

\newenvironment{innerframe}[1][]{%
	\mdfsetup{%
				innerleftmargin=5pt,%
				innerrightmargin=5pt,%
	          linewidth=0.5pt,%
	          roundcorner=5pt,%
	          backgroundcolor=white,%
	          innertopmargin=\baselineskip,
			  singleextra={
				\node[innertitle] at (P-|O)%
					{#1};
			  }
	}%
	\vspace*{-0.2cm}\begin{mdframed}[]\relax%
}%
{%
\end{mdframed}%
}

%%% Command to add a new unit definition %%%

\newcommand{\defunit}{
	\setkeys{unit}{%
		name=, profile=, cost=, costpermodel=, additionalmodels=, type=, unitsize=, basesize=, specialrules=, magiclevel=, magicpaths=, equipment=, unitequipment=, options=, mounts=, commandgroup=, unitrules=, additional=%
	}%
	\setkeys{unit}%
}

\newcommand{\showunit}[1]{
	\defunit{#1}
	\begin{unitframeFlot}[!htbp]
	\begin{unitframe}[\unit@name]{\unit@cost}
	\mdfsetup{style=defaultoptions}
	\vspace*{0.25cm}
	\begin{mdframed}
		\expandafter\profile\expandafter{\unit@profile}
	\end{mdframed}
	\vspace*{-0.2cm}
	\setlength\multicolsep{0pt}
	\begin{multicols}{2}
		% \raggedcolumns	
		\unitsize{\unit@unitsize} \par
		\specialrules{\unit@specialrules} \par
		\magic{\unit@magiclevel}{\unit@magicpaths} \par
		\equipment{\unit@equipment} \par
		\expandafter\ifblank\expandafter{\unit@additionalmodels}{}{\preto{\unit@options}{\Upto{ }\unit@additionalmodels\/ \labels@additionalmodels =\permodel < \unit@costpermodel,}}
		\options{\unit@options} \par
		\mounts{\unit@mounts} \par
		\expandafter\ifblank\expandafter{\unit@commandgroup}{}{\expandafter\commandgroup\expandafter{\unit@commandgroup} \par}
		\unitrules{\unit@unitrules} \par
		\unitequipment{\unit@unitequipment} \par
	\end{multicols}
	\unit@additional \par
	\end{unitframe}
	\end{unitframeFlot}
}
